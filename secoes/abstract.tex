\begin{resumo}[Abstract]
 \begin{otherlanguage*}{english}
  
The interest in robot programming environments for educational purposes has grown in recent years. However, these environments do not have automated mechanisms to verify the programs, which makes it impossible for students and teachers to have fast and automatic feedback on the operation of the programs. Model Verification is a software engineering technique where systems are specified in a formal language for the purpose of verifying the properties, thus, through a model checking, formally proving the absence of problems. In this work, programs written in the programming language for the simulation of robots called ROBO are automatically translated into a formal specification called CSP (Communicating Sequential Processes) through an automated translation approach, where later properties are verified in a model checking called FDR (Failures-Divergences Refinement). The translation was implemented through a platform called Spoofax, where we defined the ROBO language grammar and specified ROBO transformation rules for CSP. In addition, we validate the approach using a tool to verify the behavior of a ROBO program with variables and procedures.

   \vspace{\onelineskip}
 
   \noindent 
   \textbf{Keywords}:  Software Engineer, Software Verification, Formal Methods, Model Checking, CSP, Automatic Translation, Spoofax.
 \end{otherlanguage*}
\end{resumo}