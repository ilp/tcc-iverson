\chapter{Conclusão}
\label{chap:cap5}
%escrever minha conclusao

\section{Trabalhos Relacionados}

O único trabalho encontrado na literatura que realiza verificação de programas de robôs educacionais de forma totalmente automática é~\cite{nogueira}. Porém encontramos outro trabalho que usa verificação automática no contexto educacional. O artigo~\cite{SVA} propõe SVA (\textit{Shared Variable Programming}), que é uma ferramenta para o aprendizado de programação concorrente usando uma linguagem educacional; esta ferramenta integra um compilador da linguagem educacional para um modelo CSP com o verificador de modelos FDR. SVA permite analisar propriedades de programas concorrentes como, por exemplo, se os programas entram ao mesmo tempo em uma região crítica. Além disso, a ferramenta possui uma interface gráfica que apresenta os resultados das verificações retornados por FDR de uma forma amigável para o usuário. Esse trabalho mostra que é possível utilizar o FDR e CSP para a criação de ferramentas de verificação automática de linguagens de domínio específico. 

Em \cite{emanuel2017}, é proposto um protótipo de um ambiente para avaliação automática de robôs virtuais. Esse trabalho é uma continuação do trabalho \cite{nogueira} com foco no projeto e prototipação da interface gráfica de um ambiente para avaliação automática de robôs virtuais. O ambiente proposto objetiva a avaliação de forma automática e \textit{feedback} sobre o funcionamento dos programas escritos em ROBO, através de uma interface gráfica que lembra um sistema de julgamento online (\textit{Online Judgment System}) utilizados nas maratonas de programação. A implementação deste ambiente depende diretamente de um dos produtos deste trabalho, que é a extensão do compilador de ROBO para CSP e sua integração com a ferramenta FDR.

O trabalho \cite{silva} apresenta um método para a verificação automática durante a simulação de futebol de robôs, esse trabalho considera a especificação formal e a verificação de planos de um time de robôs simulados. A simulação ocorre de modo que vários robôs estão executando ao mesmo tempo e propondo uma solução conjunta, onde o autor denomina como sistemas multiagente. Nesse trabalho é utilizado um verificador de modelos para analisar algumas propriedades, dentre elas estão a verificação de ausência de \textit{deadlocks} e/ou \textit{livelocks} levando em consideração o tempo, pois o tempo é um fator essencial a ser considerado durante a verificação da simulação de futebol de robôs.  Por este motivo é utilizado o verificador de modelos chamado UPPAAL que oferece um conjunto de ferramentas para a verificação de sistemas em tempo real utilizando autômatos, além de utilizar uma linguagem chamada TCTL (Lógica de Árvore de Cálculo Temporizado) para formalizar as propriedades esperadas para o sistema. A diferença desse trabalho mostrado para esta proposta de pesquisa é que a verificação automática considera o tempo e sua abordagem foi desenvolvida especificamente para sistemas onde vários agentes atuam simultaneamente e por isso faz uso de uma outra ferramenta de verificação diferente da FDR.
	
\citeonline{SSS147734} propõe a verificação formal de robôs para assistência pessoal. Esses robôs estão presentes nas casas das pessoas e as ajudam em suas tarefas diárias, e como se trata de interação real com pessoas é necessária uma garantia de segurança, para que esses robôs não causem nenhum dano ou se coloquem em situações inesperadas. Por isso, é proposto um método de verificação formal. Essa proposta utiliza um verificador de modelos chamado SPIN que é bastante utilizado em sistemas de missões espaciais, telecomunicação e engenharias. A linguagem desse verificador é chamada PROMELA. Também foi utilizada uma linguagem chamada Brahms para modelar o robô. O processo de verificação ocorre quando o modelo escrito em Brahms é traduzido para linguagem PROMELA, onde o verificador de modelos pode verificar se as propriedades são satisfeitas ou não. Um exemplo de propriedade é saber se o robô irá se mover para cozinha quando o usuário envia o comando ao robô.  Essas propriedades são baseadas e pensadas na simulação e validação do usuário final, com o objetivo de aumentar a segurança prática e a confiabilidade dos assistentes robotizados, já que se trata de robôs físicos que estão trabalhando com pessoas reais. O processo de verificação é similar a esta proposta de pesquisa em alguns pontos, um deles é a tradução da linguagem modelada de um robô para uma especificação formal através de um compilador, contudo essa abordagem se trata de uma simulação voltada para o mundo real, enquanto a proposta deste trabalho é para ambientes educacionais de simulação de robôs virtuais, ainda mais que a DSL (Brahms) possui um propósito totalmente diferente da linguagem ROBO.

Outros trabalhos relacionados são: \cite{this-is-boogie-2-2} onde é proposto Boogie, uma linguagem desenvolvida para gerar condições ou propriedades de verificação de outras linguagens de programação para que possam ser utilizadas por verificadores de programas da linguagem de origem; e \cite{Kats} no qual é discutido os desafios e oportunidades de pesquisa para a criação de ambientes de programação com a finalidade de disponibilizá-los na Web, através de um estudo de implementação de uma DSL e um ambiente de programação. Portanto, trabalhos relacionados existem, porém, não são voltados para a verificação automática de robôs educacionais de ambientes voltados para o aprendizado de programação.

Assim sendo, este trabalho de pesquisa é de grande contribuição para o contexto educacional. Uma vez que oferece para estudantes e professores, que utilizam ambientes de programação de robôs educacionais, um mecanismo de verificação automático efetivo para ajudá-los a obter \textit{feedback} sobre a corretude dos seus programas escritos nesses ambientes. Além de contribuir também para área da Engenharia de Software através da verificação automática de sistemas educacionais utilizando uma abordagem de tradução automatizada. Portanto, esta proposta oferece grande relevância para a ciência.


\section{Trabalhos Futuros}

Os trabalhos futuros podem seguir como melhorias... 

Uma das melhorias que pode ser realizada está na criação de regras de tradução que na especificação formal seja proposta uma abordagem que simule uma memória local, dessa forma, mantém apenas as variáveis e parâmetros no escopo de um procedimento. Assim, descartando a adição destes à memória global da especificação, como foi proposto neste trabalho.

Outro caminho a seguir, seria uma extensão do compilador para realizar a checagem semântica de ROBO, como por exemplo, checagem de nomes e tipos. Atualmente, a avaliação semântica de ROBO não é realizada pelo compilador, pois espera-se que os programas tenham a semântica validada no ambiente RoboMind antes da tradução para CSP.