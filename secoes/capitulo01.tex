\chapter{Introdução}

O interesse da população pela robótica educacional vem aumentando nas últimas
décadas oferecendo benefícios em todos os níveis da educação, seja no ensino de crianças, adolescentes ou adultos (ALIMISIS, 2013). Um outro aspecto observado na literatura é que muitas aplicações de robótica são utilizadas para ajudar no aprendizado da matemática, ciência ou engenharia através de atividades práticas que envolvem a programação de robôs (BENITTI, 2012). A programação de robôs educacionais oferece apoio para o ensino e aprendizagem de programação, principalmente para aqueles que estão iniciando a construção do pensamento computacional, com o propósito de usar o raciocínio lógico para estruturar soluções coerentes para problemas complexos (BOMBASAR et. al, 2015).

Ambientes de programação de robôs são utilizados para a criação de atividades com
este fim, onde os ambientes são geralmente atrativos e lúdicos, assim pode despertar o fascínio e a curiosidade dos estudantes pela programação (SILVA et. al, 2014). Segundo o jornal Folha de Pernambuco (2016), o recente do uso desses ambientes de robótica na sala de aula como forma de apoio nas escolas públicas do estado de Pernambuco tem aumentado o interesse dos estudantes pelas disciplinas de matemática e física, ainda afirma que o uso efetivo da robótica educacional aumenta a criatividade, sociabilidade, a concentração e o senso de coletividade.

O ambiente de programação para o LEGO MindStorms é um exemplo de aplicação
com essa finalidade; apresenta uma interface para que o aluno possa escrever seus
programas e executar o programa em robôs educacionais. Um problema no uso da robótica educacional é o alto custo para adquirir equipamentos, como por exemplo, o robô MindStorms da LEGO. Uma alternativa para o uso de robôs é utilizar ambientes de robótica educacional baseados em simulação como o RoboMind. Estes ambientes permitem a simulação passo a passo do robô, a partir da execução dos comandos do programa,
ocasionando em uma melhor compreensão para o aluno (LESSA et al., 2015).

Um problema dos ambientes de simulação de robôs é que estes não oferecem um
mecanismo de verificação automática das soluções propostas por estudantes, ou seja, os programas escritos pelos alunos não são verificados quanto a sua corretude para o problema proposto, de modo que estudante e professor possam obter feedback automático sobre o funcionamento dos programas. O atual mecanismo de verificação em ambientes de simulação de robôs virtuais ocorre através da observação dos passos do robô, o que pode tornar uma tarefa demorada e trabalhosa. Além de onerosa, a verificação pode ser complexa. Por isso, métodos de verificação automática são tão importantes para determinar a corretude de um programa em diferentes perspectivas de maneira automatizada e rápida (DUARTE, 2011). 

Portanto, este trabalho propõe uma abordagem para a tradução automática de uma
linguagem de programação de robôs para um modelo formal com o objetivo de fornecer a
alunos e professores uma forma de apoio durante a programação nesses ambientes por meio de feedback automático. Além disso, este trabalho contribui para a área da Engenharia de Software através da junção da educação com a verificação de sistemas de software, por meio da abordagem proposta para resolver o problema descrito em seguida.
