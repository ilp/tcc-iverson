\setlength{\absparsep}{18pt} % ajusta o espaçamento dos parágrafos do resumo
\begin{resumo}
 
Ambientes de programação de robôs com o intuito educacional têm crescido muito nos últimos anos. No entanto, esses ambientes não possuem mecanismos automatizados para verificação dos programas escritos, o que impossibilita que alunos e professores tenham um \textit{feedback} rápido sobre o funcionamento dos programas. Técnicas da Engenharia de Software são utilizadas para realizar verificação automática de software. É o caso da técnica Verificação de Modelos, onde sistemas são especificados em uma linguagem formal com a finalidade de verificar diferentes propriedades, assim, através de um verificador de modelos, provando formalmente a ausência de problemas. 
%Diante disso, programas precisam ser modelados em uma linguagem formal para que seja possível verificar automaticamente através um verificador de modelos.
Neste trabalho, programas escritos na linguagem de programação para a simulação de robôs chamada ROBO são traduzidos para uma notação formal chamada CSP (\textit{Communicating Sequential Processes}) onde propriedades são verificadas em um verificador de modelos chamado FDR (\textit{Failures-Divergences Refinement}). Desse modo, este trabalho propõe uma abordagem para a tradução automática da linguagem ROBO para o modelo formal CSP possibilitando a alunos e professores uma forma de apoio durante a programação nesses ambientes por meio de \textit{feedback} automático. Além disso, 
%Desse modo, este trabalho tem por objetivo propor uma abordagem de tradução automática de uma linguagem de programação de robôs para um modelo formal.

 \textbf{Palavras-chave}: Engenharia de Software, Verificação de Software, Métodos Formais, Verificador de Modelos, CSP, Tradução Automática, Spoofax.
\end{resumo}

