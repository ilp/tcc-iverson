\setlength{\absparsep}{18pt} % ajusta o espaçamento dos parágrafos do resumo
\begin{resumo}
 
O interesse por ambientes de programação de robôs virtuais para fins educacionais têm crescido nos últimos anos. Estes ambientes são uma alternativa para o uso de robôs reais que possuem um alto custo de aquisição. No entanto, não existem ambientes gratuitos que oferecem  mecanismos automatizados para verificação dos programas de robôs virtuais, o que impossibilita que alunos e professores tenham um \textit{feedback} rápido e automático sobre o funcionamento dos programas. 
Este trabalho propõe uma abordagem de verificação automática de programas de robôs virtuais escritos na linguagem educacional ROBO. Desenvolvemos um compilador que lê programas escritos em ROBO e traduz os programas para uma notação formal chamada CSP (\textit{Communicating Sequential Processes}), que é a entrada para uma ferramenta de verificação automática de modelos chamada FDR (\textit{Failures-Divergences Refinement}). As fases da compilação foram implementadas usando a plataforma Spoofax, onde definimos a gramática da linguagem ROBO e especificamos regras de tradução de ROBO para CSP.  
Este trabalho remove uma limitação da nossa abordagem anterior de verificação que não permite a análise de programas ROBO contendo variáveis e procedimentos.
Uma importante contribuição deste trabalho é a extensão da abordagem de verificação para permitir a análise automática de programas ROBO com variáveis e procedimentos. 
A extensão consiste na modificação da gramática do compilador pela inclusão de variáveis e procedimentos e na inclusão de novas regras de tradução que definem a semântica formal para os elementos adicionados na gramática.
O trabalho propõe uma ferramenta que torna transparente o processo de tradução de ROBO para CSP e a verificação automática usando FDR. Validamos a abordagem utilizando a ferramenta proposta para verificar o comportamento de um programa ROBO com variáveis e procedimentos.


%Desse modo, este trabalho propõe uma abordagem para a tradução automática da linguagem ROBO para o modelo formal CSP possibilitando a alunos e professores uma forma de apoio durante a programação nesses ambientes por meio de \textit{feedback} automático.
%Desse modo, este trabalho tem por objetivo propor uma abordagem de tradução automática de uma linguagem de programação de robôs para um modelo formal.

 \textbf{Palavras-chave}: Engenharia de Software, Verificação de Software, Métodos Formais, Verificador de Modelos, CSP, Tradução Automática, Spoofax.
\end{resumo}

