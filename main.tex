\documentclass[
	openany,
	% -- opções da classe memoir --
	12pt,				% tamanho da fonte
   % openright,
	%twoside,
    oneside,
    % para impressão em verso e anverso. Oposto a oneside
	a4paper,			% tamanho do papel. 
	brazil				% o último idioma é o principal do documento
	]{abntex2}

% ---
% Pacotes básicos 
% ---

\usepackage{fontspec}
\setmainfont{Arial}

\usepackage[utf8]{inputenc}		% Codificacao do documento (conversão automática dos acentos)
\usepackage{indentfirst}		% Indenta o primeiro parágrafo de cada seção.
\usepackage{listings}
\usepackage{color}% Controle das cores
\usepackage{graphicx}			% Inclusão de gráficos
\usepackage{microtype} 			% para melhorias de justificação
\usepackage{multicol}			% multiplas colunas no texto
\usepackage{subcaption}
\usepackage{caption}
\usepackage{float}
\usepackage{amsmath}
\usepackage{amssymb}
\usepackage{amsthm}
\usepackage{lipsum}
\usepackage{blindtext}
\usepackage{varwidth}
\usepackage{fancyvrb}
\usepackage{tcolorbox}
\usepackage{numberedblock}


\newcommand*\justify{%
  \fontdimen2\font=0.4em% interword space
  \fontdimen3\font=0.2em% interword stretch
  \fontdimen4\font=0.1em% interword shrink
  \fontdimen7\font=0.1em% extra space
  \hyphenchar\font=`\-% allowing hyphenation
}

\definecolor{eclipseBlue}{RGB}{42,0.0,255}
\definecolor{eclipseGreen}{RGB}{63,127,95}
\definecolor{eclipsePurple}{RGB}{127,0,85}
\lstdefinelanguage{Robo}
{
    keywords=[1]{
    },
    keywordstyle=[1]\color{eclipsePurple},
    keywords=[2]{},
    keywordstyle=[2]\color{eclipseBlue},
    sensitive=false,
    morestring=[b]',
    morecomment=[l]{--}
}
\lstset{
  language={Robo},
  basicstyle=\linespread{0.8}\ttfamily, % Global Code Style
  captionpos=t, % Position of the Caption (t for top, b for bottom)
  extendedchars=true, % Allows 256 instead of 128 ASCII characters
  tabsize=2, % number of spaces indented when discovering a tab 
  columns=fixed, % make all characters equal width
  keepspaces=true, % does not ignore spaces to fit width, convert tabs to spaces
  showstringspaces=false, % lets spaces in strings appear as real spaces
  breaklines=true, % wrap lines if they don't fit
  frame=trbl, % draw a frame at the top, right, left and bottom of the listing
  frameround=tttt, % make the frame round at all four corners
  framesep=4pt, % quarter circle size of the round corners
  numbers=left, % show line numbers at the left
}

\captionsetup[table]{labelsep=space, 
         justification=raggedright, singlelinecheck=off}

% ---
% ---
% Formatação de código-fonte
% ---
\usepackage{listings}

% Altera o nome padrão do rótulo usado no comando \autoref{}
\renewcommand{\lstlistingname}{Código}

% Altera o rótulo a ser usando no elemento pré-textual "Lista de código"
\renewcommand{\lstlistlistingname}{Lista de códigos}

% Configura a ``Lista de Códigos'' conforme as regras da ABNT (para abnTeX2)
\begingroup\makeatletter
\let\newcounter\@gobble\let\setcounter\@gobbletwo
  \globaldefs\@ne \let\c@loldepth\@ne
  \newlistof{listings}{lol}{\lstlistlistingname}
  \newlistentry{lstlisting}{lol}{0}
\endgroup

\renewcommand{\cftlstlistingaftersnum}{\hfill--\hfill}

\let\oldlstlistoflistings\lstlistoflistings
\renewcommand{\lstlistoflistings}{%
   \begingroup%
   \let\oldnumberline\numberline%
   \renewcommand{\numberline}{\lstlistingname\space\oldnumberline}%
   \oldlstlistoflistings%
   \endgroup}
   
 %-----

% Pacotes de citações
% ---
\usepackage[brazilian]{backref}	 % Paginas com as citações na bibl
\usepackage[alf]{abntex2cite}	% Citações padrão ABNT

% --- 
% CONFIGURAÇÕES DE PACOTES
% --- 

% ---
% Configurações do pacote backref
% Usado sem a opção hyperpageref de backref
\renewcommand{\backrefpagesname}{Citado na(s) página(s):~}
% Texto padrão antes do número das páginas
\renewcommand{\backref}{}
% Define os textos da citação
\renewcommand*{\backrefalt}[4]{
	\ifcase #1 %
		Nenhuma citação no texto.%
	\or
		Citado na página #2.%
	\else
		Citado #1 vezes nas páginas #2.%
	\fi}%
% ---

% ---
% Informações de dados para CAPA e FOLHA DE ROSTO
% ---
\titulo{Uma Abordagem para Tradução de uma Linguagem de Programação de Robôs para um Modelo Formal}
\autor{Iverson Luís Pereira}
\local{Recife}
\data{Agosto, 2018}
\orientador{Professor Sidney de Carvalho Nogueira}


\instituicao{%
	Universidade Federal Rural de Pernambuco -- UFRPE
  	\par
  	Departamento de Computação
    \par
  	Curso de Bacharelado em Ciência da Computação
}

\tipotrabalho{Trabalho de Conclusão de Curso}
% O preambulo deve conter o tipo do trabalho, o objetivo, 
% o nome da instituição e a área de concentração 
\preambulo{Monografia apresentada ao Curso de Bacharelado em Ciência da Computação da Universidade Federal Rural de Pernambuco, como requisito parcial para obtenção do título de Bacharel em Ciência da Computação.}
% ---


% ---
% Configurações de aparência do PDF final

% alterando o aspecto da cor azul
\definecolor{blue}{RGB}{41,5,195}

% informações do PDF
\makeatletter
\hypersetup{
     	%pagebackref=true,
		pdftitle={\@title}, 
		pdfauthor={\@author},
    	pdfsubject={\imprimirpreambulo},
	    pdfcreator={LaTeX with abnTeX2},
		colorlinks=true,       		% false: boxed links; true: colored links
    	linkcolor=blue,          	% color of internal links
    	citecolor=blue,        		% color of links to bibliography
    	filecolor=magenta,      		% color of file links
		urlcolor=blue,
		bookmarksdepth=4
}
\makeatother
% --- 

% --- 
% Espaçamentos entre linhas e parágrafos 
% --- 

% O tamanho do parágrafo é dado por:
\setlength{\parindent}{1.3cm}

% Controle do espaçamento entre um parágrafo e outro:
\setlength{\parskip}{0.2cm}  % tente também \onelineskip

% ---
% compila o indice
% ---
\makeindex
% ---

% ----
% Início do documento
% ----
\begin{document}

% Seleciona o idioma do documento (conforme pacotes do babel)
%\selectlanguage{english}
\selectlanguage{brazil}

% Retira espaço extra obsoleto entre as frases.
\frenchspacing 

% ----------------------------------------------------------
% ELEMENTOS PRÉ-TEXTUAIS
% ----------------------------------------------------------
% \pretextual
%\begin{figure}[h]
%\centering % este comando é usado para centralizar a figura
%\includegraphics[width=7cm]{figuras/logo_ufrpe_horizontal.png}\\
%\end{figure}

% \begin{figure}[ht]
% \centering
% \begin{minipage}[b]{0.45\textwidth}
% \includegraphics[height=3cm]{figuras/logo_ufrpe_horizontal.png}
% \end{minipage}
% \qquad
% \begin{minipage}[b]{0.45\textwidth}
% \includegraphics[height=2.5cm]{figuras/logo_bsi.pdf}
% \end{minipage}
% \end{figure}

%\begin{minipage}[t]{1\textwidth}
	\begin{figure}[ht]
	    \hspace{4.5cm}
		\includegraphics[height=3cm]{figuras/logo_ufrpe_horizontal.png}
	\end{figure}    
%\end{minipage}

% ---
% Capa
% ---
\imprimircapa
% ---
% ---
% Folha de rosto
% (o * indica que haverá a ficha bibliográfica)
% ---
\imprimirfolhaderosto
% ---

% dedicatoria
\begin{dedicatoria}
   \vspace*{\fill}
   \centering
   \noindent
   \textit{À meu pai.\\} \vspace*{\fill}
\end{dedicatoria}

% agradecimentos
\begin{agradecimentos}
Agradeço à minha mãe, Rosânia, por sempre me apoiar e me incentivar nas horas mais difíceis. Obrigado por sua dedicação à família. Você é o meu maior tesouro, a minha maior fortaleza. Te amo, minha rainha.

Agradeço ao meu pai, Ivanildo, que tanto lutou e possibilitou a nossa família uma vida melhor na cidade grande. Onde, pude ter a oportunidade de ir a escola e mais tarde à Universidade. Tenho muito orgulho de você pai, te amo!

Agradeço ao meu grande e único irmão, Igor, a quem tenho muito carinho, e sua esposa Naiany por sempre estarem ao meu lado.

Agradeço aos meus avós, tias e primos(as), que de alguma forma contribuíram em minha jornada e compreenderam minha ausência pelo tempo dedicado aos estudos.

Aos amigos da vida pela força e torcida para que tudo desse certo. Especialmente, Artur, Igor, Kamila, Mariana e Osmar. Vocês são os irmãos que a vida me deu.

Ao meu orientador, Sidney Nogueira, pela dedicação, pela paciência, por acreditar e motivar nosso trabalho, pelo apoio constante e pela atenção durante todo o período de construção deste trabalho. A quem tenho grande admiração e respeito. Meu muito obrigado!

Ao meu tutor, Mike, e ao grupo PET-Ciranda da Ciência, por me incentivar nas atividades ensino, pesquisa e extensão, atividades essenciais durante minha trajetória acadêmica.

Meus sinceros agradecimentos a Universidade Federal Rural de Pernambuco e a todos os professores do curso de Bacharelado em Ciência da Computação. Aos meus amigos e companheiros de curso, em especial, Ana Juriti, Bruno Marques, Carlos Magnum, Danielly Queiroz, Jeremias Leite e Ricardo Luna. Algumas das pessoas mais inteligentes e fascinantes que tive a oportunidade de conviver.

Agradeço ao Núcleo de Tecnologia da Informação - NTI/UFPE que me deu a oportunidade de aplicar na prática todo o conhecimento que adquiri durante a minha graduação, além de aprender com colegas de trabalho incríveis.

Finalmente, agradeço à Deus por ter me dado forças e me fazer querer continuar. Sou extremamente grato à Ele, por ter me proporcionado momentos gloriosos com pessoas essenciais à minha vida e conquistado grandes oportunidades.


\end{agradecimentos}



% epigrafe
\begin{epigrafe}
    \vspace*{\fill}
	\begin{flushright}
		\textit{``Isn't it splendid to think of all the things there are to find out about? It just makes me feel glad to be alive -- it's such an interesting world. It wouldn't be half so interesting if we know all about everything, would it? There'd be no scope for imagination then, would there?'' \\
		(L.M. Montgomery, Anne of Green Gables)}
	\end{flushright}
\end{epigrafe}

% resumo e abstract
\setlength{\absparsep}{18pt} % ajusta o espaçamento dos parágrafos do resumo
\begin{resumo}
 
O interesse por ambientes de programação de robôs virtuais para fins educacionais têm crescido nos últimos anos. Estes ambientes são uma alternativa para o uso de robôs reais que possuem um alto custo de aquisição. No entanto, não existem ambientes gratuitos que oferecem  mecanismos automatizados para verificação dos programas de robôs virtuais, o que impossibilita que alunos e professores tenham um \textit{feedback} rápido e automático sobre o funcionamento dos programas. 
Este trabalho propõe uma abordagem de verificação automática de programas de robôs virtuais escritos na linguagem educacional ROBO. Desenvolvemos um compilador que lê programas escritos em ROBO e traduz os programas para uma notação formal chamada CSP (\textit{Communicating Sequential Processes}), que é a entrada para uma ferramenta de verificação automática de modelos chamada FDR (\textit{Failures-Divergences Refinement}). As fases da compilação foram implementadas usando a plataforma Spoofax, onde definimos a gramática da linguagem ROBO e especificamos regras de tradução de ROBO para CSP.  
Este trabalho remove uma limitação da nossa abordagem anterior de verificação que não permite a análise de programas ROBO contendo variáveis e procedimentos.
Uma importante contribuição deste trabalho é a extensão da abordagem de verificação para permitir a análise automática de programas ROBO com variáveis e procedimentos. 
A extensão consiste na modificação da gramática do compilador pela inclusão de variáveis e procedimentos e na inclusão de novas regras de tradução que definem a semântica formal para os elementos adicionados na gramática.
O trabalho propõe uma ferramenta que torna transparente o processo de tradução de ROBO para CSP e a verificação automática usando FDR. Validamos a abordagem utilizando a ferramenta proposta para verificar o comportamento de um programa ROBO com variáveis e procedimentos.


%Desse modo, este trabalho propõe uma abordagem para a tradução automática da linguagem ROBO para o modelo formal CSP possibilitando a alunos e professores uma forma de apoio durante a programação nesses ambientes por meio de \textit{feedback} automático.
%Desse modo, este trabalho tem por objetivo propor uma abordagem de tradução automática de uma linguagem de programação de robôs para um modelo formal.

 \textbf{Palavras-chave}: Engenharia de Software, Verificação de Software, Métodos Formais, Verificador de Modelos, CSP, Tradução Automática, Spoofax.
\end{resumo}


\begin{resumo}[Abstract]
 \begin{otherlanguage*}{english}
  
The interest in robot programming environments for educational purposes has grown in recent years. However, these environments do not have automated mechanisms to verify the programs, which makes it impossible for students and teachers to have fast and automatic feedback on the operation of the programs. Model Verification is a software engineering technique where systems are specified in a formal language for the purpose of verifying the properties, thus, through a model checking, formally proving the absence of problems. In this work, programs written in the programming language for the simulation of robots called ROBO are automatically translated into a formal specification called CSP (Communicating Sequential Processes) through an automated translation approach, where later properties are verified in a model checking called FDR (Failures-Divergences Refinement). The translation was implemented through a platform called Spoofax, where we defined the ROBO language grammar and specified ROBO transformation rules for CSP. In addition, we validate the approach using a tool to verify the behavior of a ROBO program with variables and procedures.

   \vspace{\onelineskip}
 
   \noindent 
   \textbf{Keywords}:  Software Engineer, Software Verification, Formal Methods, Model Checking, CSP, Automatic Translation, Spoofax.
 \end{otherlanguage*}
\end{resumo}

% ---
% inserir lista de ilustrações
% ---
\pdfbookmark[0]{\listfigurename}{lot}
\begin{KeepFromToc}
\listoffigures
\end{KeepFromToc}
\cleardoublepage
% ---

% ---
% inserir lista de tabelas
% ---
\pdfbookmark[0]{\listtablename}{lot}
\listoftables*
\cleardoublepage
% ---

% ---
% inserir lista de listings
% ---
\pdfbookmark[0]{\lstlistlistingname}{lol}
\begin{KeepFromToc}
\lstlistoflistings
\end{KeepFromToc}
\cleardoublepage
% ---

% ---
% inserir lista de abreviaturas e siglas
% ---
\begin{siglas}
  \item[ATerm] Annotated Term Format
  \item[AST] Abstract Syntax Tree
  \item[CSP] Communicating Sequential Processes	
  \item[FDR] Failures-Divergences Refinement
  \item[DSL] Domain-Specific Language
  \item[...] \ldots
\end{siglas}
% ---

% ---
% inserir o sumario
% ---
\pdfbookmark[0]{\contentsname}{toc}
\tableofcontents*
\cleardoublepage
% ---

% ----------------------------------------------------------
% ELEMENTOS TEXTUAIS
% ----------------------------------------------------------
\textual

% ----------------------------------------------------------
% inclusao das secoes do texto
% ----------------------------------------------------------
\chapter{Introdução}

O interesse pela robótica educacional vem aumentando nas últimas décadas, uma vez que a mesma trás benefícios em todos os níveis da educação, seja no ensino de crianças, adolescentes ou adultos \cite{alimisis}. Robótica tem sido utilizada para ajudar no aprendizado da matemática, ciências ou engenharia através de atividades práticas que envolvem a programação de robôs \cite{Benitti2012}. A programação de robôs educacionais oferece apoio para o ensino e aprendizagem de programação, principalmente para aqueles que estão iniciando a construção do pensamento computacional, com o propósito de usar o raciocínio lógico para estruturar soluções coerentes para problemas complexos \cite{Bombasar2015}.

Ambientes de programação de robôs são geralmente atrativos e lúdicos com a intenção de despertar o fascínio e a curiosidade dos estudantes pela programação \cite{ESILVA}. Segundo a reportagem~\citeonline{jornal}, o uso de ambientes de robótica nas escolas públicas do estado de Pernambuco tem aumentado o interesse dos estudantes pelas disciplinas de matemática e física, além de aumentar a criatividade, a sociabilidade, a concentração e o senso de coletividade.

O ambiente de programação LEGO MindStorms\footnote[1]{https://www.lego.com/en-us/mindstorms} é um exemplo de aplicação com essa finalidade; apresenta uma interface onde o aluno pode desenvolver o programa que controla o robô antes de executar o programa no robô. Um problema no uso da robótica educacional é o alto custo para adquirir equipamentos, como por exemplo, o robô MindStorms da LEGO. Uma alternativa para o uso de robôs é utilizar ambientes de robótica educacional baseados em simulação como o RoboMind\footnote[2]{http://robomind.net/}. Estes ambientes permitem a simulação passo a passo do robô na tela do computador, a partir da execução dos comandos do programa, ocasionando em uma melhor compreensão para o aluno \cite{Lessa2015}. 

Um problema dos ambientes de simulação de robôs é que estes não oferecem um mecanismo de verificação automática das soluções propostas por estudantes, ou seja, os programas escritos pelos alunos não são verificados quanto a sua corretude para o problema proposto, de modo que estudante e professor possam obter \textit{feedback} automático sobre o funcionamento dos programas. O atual mecanismo de verificação em ambientes de simulação de robôs virtuais ocorre através da observação dos passos do robô, o que pode tornar a tarefa demorada e trabalhosa. Além de onerosa, a verificação pode ser complexa. Por isso, métodos de verificação automática são tão importantes para determinar a corretude de um programa em diferentes perspectivas de maneira automatizada e rápida \cite{Duarte}. 

%Portanto, este trabalho propõe uma abordagem para a tradução automática de uma linguagem de programação de robôs para um modelo formal com o objetivo de fornecer a alunos e professores uma forma de apoio durante a programação nesses ambientes por meio de \textit{feedback} automático. Além disso, este trabalho contribui para a área da Engenharia de Software através da junção da educação com a verificação de sistemas de software, por meio da abordagem proposta para resolver o problema descrito em seguida


%\section{Problema de Pesquisa}

Não existem ambientes gratuitos de programação de robôs que oferecem a estudantes e professores um mecanismo de análise automática dos programas. O único ambiente de verificação para robôs é o Robomind Academy\footnote[3]{https://www.robomindacademy.com/}. Este ambiente além de ser pago, só permite a verificação dos mapas já cadastrados no sistema, que estão associados a desafios de programação fixados; não é possível realizar a verificação automática dos programas levando em consideração diferentes mapas e soluções para um determinado problema. Exemplos de problemas são: identificar se o robô encontra a saída de um labirinto; se o robô encontra um objeto no mapa; ou simplesmente se o robô termina sua execução (não executa um laço infinito).

%O RoboMind, mostrado na Figura \ref{fig:robomind}, é um ambiente de programação de robôs virtuais para o ensino e aprendizagem de robótica; possui uma interface com um espaço para a escrita dos programas e um outro espaço onde o aluno pode acompanhar a execução do robô em um mapa. Nesse ambiente, os programas são escritos na linguagem ROBO, uma linguagem educacional desenvolvida para a programação de robôs que oferece os principais comandos de programação: estruturas condicionais, estruturas de repetição, procedimentos e declaração de variáveis. O programa ROBO, ilustrado no lado esquerdo da Figura 1, movimenta o robô enquanto o objeto (beacon) não é detectado à sua frente: se não há nenhum obstáculo à sua frente, o robô avança uma célula (forward), caso contrário o robô recua uma célula (backward) e muda sua orientação para a direita (right).


%O presente trabalho propõe a verificação automática de robôs programados na linguagem ROBO considerando diferentes mapas e desafios utilizando a técnica de verificação de modelos. Os programas considerados em ROBO possuem uma quantidade finita de estados, portanto a técnica possibilita realizar análises completas neste contexto.

%Um dos problemas para realizar verificação de modelos é que não existe modelo formal para linguagens de programação de robôs. 

Em um trabalho anterior~\cite{nogueira} propomos a sistematização do processo de tradução dos programas escritos em ROBO para um modelo formal. Naquele trabalho é utilizado a verificação de modelos para automatizar a verificação de programas de robôs virtuais no ambiente RoboMind. Esta abordagem de verificação se dá por meio de algumas etapas como mostrado na Figura~\ref{fig:fluxograma}. É usado como entrada o programa escrito na linguagem ROBO e o mapa no qual o programa será executado. Um processo de tradução automática produz como saída a especificação formal do programa escrita na linguagem da álgebra de processos CSP (\textit{Communicatting Sequential Processes})~\cite{Cleaveland2018}. Através de um processo automático também é obtida a representação formal do mapa. A especificação formal do mapa e do robô são entradas do verificador de modelos FDR~\cite{Gibson} que é utilizado para verificar as propriedades do robô. Como exemplo de propriedade, temos se o programa possui \textit{deadlock}, ou seja, se o programa termina sua execução. 

\begin{figure}[h]
\centering
\caption{Visão geral da abordagem de verificação automática}
\includegraphics[height=5cm]{figuras/approach_workflow.png}
\fonte{\cite{nogueira}}
\label{fig:fluxograma}
\end{figure} 

%Em \cite{nogueira}, a especificação formal (ou modelo formal) é construída na linguagem de álgebra de processos CSP (\textit{Communicatting Sequential Processes}). CSP é uma ótima linguagem para expressar também a execução simultânea de vários robôs em um mesmo programa. Tal possibilidade não foi explorada por trabalhos disponíveis na literatura.

A maior complexidade da abordagem ilustrada na Figura \ref{fig:fluxograma} é a tradução da notação ROBO para a notação formal de CSP. Esta tradução é definida a partir de regras de mapeamento de cada elemento da linguagem ROBO para o seu equivalente em CSP; estas regras fazem parte de um compilador que automatiza a aplicação das regras. 
%No processo de criação do compilador, primeiro, as regras de mapeamento devem ser realizadas utilizando um \textit{framework} que contemple as etapas de criação de uma Linguagem de Domínio Específico (\textit{Domain-Specific Language} - DSL). As DSLs são linguagens de alto poder de abstração linguístico onde o desenvolvedor pode focar na lógica da aplicação, o que permite que elas possam ser automatizadas, analisadas e otimizadas. 
As etapas internas do compilador são: (1) criação de um \textit{parser} para a sintaxe da linguagem ROBO e (2) a geração do modelo CSP a partir da árvore sintática do programa usando regras de mapeamento. 
%; (5) integrar a linguagem dentro de um ambiente de desenvolvimento \cite{KatsSpoofax}. 
O \textit{framework} Spoofax~\cite{KatsSpoofax} foi utilizado para facilitar a implementação das etapas internas do compilador. 
%No contexto deste trabalho, a DSL corresponde a linguagem ROBO. Para esta DSL, a compilação deve considerar toda a sintaxe da linguagem ROBO, desde comandos padrões (condicionais e repetição) até mesmo declaração de variáveis e procedimentos. Para esta pesquisa o \textit{framework} Spoofax, o mesmo mostrado no artigo anterior, é adotado para realizar o mapeamento de ROBO para CSP, pois oferece um ambiente de desenvolvimento completo durante as etapas de compilação.

Atualmente, o compilador de ROBO para CSP não considera vários elementos importantes da linguagem como variáveis e procedimentos. Além disto, não existe uma integração do compilador com o verificador de modelos FDR. Também não existe uma interface que permita ao usuário utilizar a abordagem da Figura \ref{fig:fluxograma} de forma transparente. 

O presente trabalho estende a gramática do compilador para aceitar programas ROBO com variáveis e procedimentos, implementar regras de mapeamento de ROBO (que incluem variáveis e procedimentos) para CSP usando o \textit{framework} Spoofax  e integrar o compilador com FDR através de uma ferramenta que torne transparente para o usuário o processo de tradução e verificação.

%\section{Objetivos}
%Nesta seção estão dispostos os objetivos pretendidos por esta pesquisa.
%\subsection{Objetivo Geral}
%
%O trabalho proposto tem como objetivo a extensão e a automação de uma abordagem de tradução de uma linguagem através do desenvolvimento de um compilador com finalidade de realizar verificação automática de programas de robôs educacionais por meio de um modelo formal.
%
%\subsection{Objetivos Específicos}
%
%\begin{enumerate}
%    \item Definir regras de compilação para variáveis e procedimentos;
%    \item Estender o compilador existente da linguagem de programação ROBO para notação formal CSP;
%    \item Propor um protótipo funcional para análise de programas que integra o compilador com o verificador de modelos definido para este projeto.
%
%\end{enumerate}

%\section{Organização do Trabalho}

Os demais capítulos estão dispostos da seguinte maneira: o Capítulo \ref{chap:cap2} apresenta os aspectos teóricos utilizados ao longo desta pesquisa; o Capítulo \ref{chap:cap3} expõe a principal contribuição deste trabalho; no Capítulo \ref{chap:cap4} validamos a abordagem proposta por meio de um protótipo funcional; por fim, no Capítulo \ref{chap:cap5} são apresentados alguns trabalhos relacionados, além de expressar as considerações finais e expectativas para trabalhos futuros.
\chapter{Fundamentação Teórica}
\label{chap:cap2}

\section{Métodos Formais}
\section{Verificador de Modelos}
\subsection{FDR}
\section{Linguagem de Domínio Específico}
\section{Processo de Compilação}
\subsection{Parsing}
\subsection{Transformação}
\subsection{Spoofax}
\chapter{Tradução de ROBO para CSP}
\label{chap:cap3}

Este capítulo descreve uma importante contribuição deste trabalho que é o processo de tradução automática de ROBO para CSP utilizando o \textit{framework} Spoofax. Este processo inclui a definição da sintaxe da linguagem ROBO até a definição das regras de transformação de programas ROBO para a sua representação CSP correspondente. O presente trabalho estende a abordagem de tradução atual por permitir a tradução de variáveis e procedimentos no programa ROBO para CSP.  A apresentação do processo é ilustrada através de um exemplo de programa ROBO.

% Sidney : o que a subseção a seguir diz já foi dito no capítulo de fundamentação teórica. Outra coisa é que não é um bom estilo de escrita fazer seções/subseções muito pequenas - OK
%\subsection{Ferramentas e Ambiente de Programação}
%O desenvolvimento de um compilador exige uma preparação bem elaborada de todo um ambiente de programação. No caso deste trabalho, foi necessário o uso do ambiente de programação Eclipse juntamente com um plugin do Spoofax. O qual foi essencial para o desenvolvimento da abordagem de tradução automática. O plugin tem todas as depedências para a geração de Árvore de Análise Sintática e transformação de código.

\section{Definição da Sintaxe}
% Sidney : o texto comentado a seguir já deve estar sendo dito no referencial teórico (se não estiver acrescentar lá). No capítulo de contribuição não precisa repetir.
%Essa é a etapa inicial para a construção do compilador, na qual devemos primeiro definir todos os aspectos sintáticos da linguagem de programação utilizada no ambiente RoboMind. Ou seja, essa etapa deverá ser capaz de considerar os programas escritos na linguagem ROBO e assegurá-los que estão sintaticamente corretos. 
%Para a definição da gramática livre de contexto da linguagem ROBO foi utilizado o formalismo SDF3 introduzido no Capítulo \ref{chap:cap2}. Além de definir a gramática, SDF3 também foi utilizado para especificar os símbolos (\textit{tokens}) da linguagem ROBO, como por exemplo, palavras reservadas. A partir da sintaxe em SDF3, o \textit{framework} \textit{Spoofax} produz uma Árvore Sintática Abstrata (\textit{Abstract Syntax Tree - AST}, em inglês) para um programa ROBO dado como entrada. Esta árvore será utilizada para a geração da especificação CSP usando Stratego.

No trabalho anterior~\cite{nogueira}, como já mencionado, propomos um compilador que contempla a tradução de programas ROBO sem variáveis e procedimentos para CSP. Esta seção mostra a extensão da gramática introduzida em~\cite{nogueira} para permitir que programas ROBO com variáveis e procedimentos possam ser traduzidos para CSP. 

Antes de estender a gramática SDF3 introduzida em~\cite{nogueira}, foi preciso reescrever algumas de suas produções sintáticas para simplificar a AST obtida como resultado do parsing. Esta simplificação foi fundamental para a criação das regras Stratego que lidam com variáveis e procedimentos. A gramática definida em~\cite{nogueira} resulta em uma AST onde o programa é uma sequência que possui a estrutura recursiva (\texttt{Instr}, \texttt{Sequence}) semelhante a uma árvore, o que impossibilitava o uso de funções do \textit{framework} Spoofax como filtros que trabalham com sequencias. Para facilitar o mapeamento da AST para CSP, foi necessário reconstruir partes da gramática de modo que a AST gerada apresentasse um formato de lista, o que torna possível o uso das funções nativas do Spoofax. A Figura~\ref{fig:gramatica_antes} mostra um trecho da gramática definida no trabalho anterior no formato SDF3. 

\begin{figure}[h]
\caption{Gramática escrita em forma de árvore}
\lstinputlisting[nolol]{codes/gramatica_antes.sdf3}
\fonte{O autor}
\label{fig:gramatica_antes}
\end{figure}

A Figura~\ref{fig:gramatica} mostra um fragmento da gramática após a reescrita. Na parte superior da figura, são importados quatro módulos que contém definições da gramática: o módulo \texttt{Commom} que contém toda a parte léxica da linguagem, como por exemplo, palavras reservadas; o módulo \texttt{ExpressionsBoolean} que possui todas as definições da gramática para expressões booleanas; o módulo \texttt{ExpressionsMath} que contém a sintaxe de expressões aritméticas; e o módulo \texttt{Robo2CSP} que importa os demais módulos e define a sintaxe da linguagem ROBO. Nesta gramática, o que antes era \texttt{Sequence} tornou-se \texttt{Statement} seguido pelo operador * para representar zero ou mais ocorrências. Dessa forma, uma ocorrência de \texttt{Sequence} é adicionada ao lado de outra \texttt{Sequence} e forma uma sequência. A gramática completa e os módulos auxiliares podem ser encontrados no Apêndice~\ref{apendice1}.

\begin{figure}[h]
\caption{Gramática proposta para ROBO}
\lstinputlisting[nolol]{codes/gramatica.sdf3}
\fonte{O autor}
\label{fig:gramatica}
\end{figure}


A partir da gramática reescrita foram acrescentadas produções sintáticas que correspondem a variáveis e procedimentos. A primeira parte da extensão consistiu na adição do termo \texttt{Declaration} (linha 18 na Figura~\ref{fig:gramatica}) que possui três tipos: \texttt{Variable}, \texttt{Procedure} e \texttt{ProcParam} (linhas 20, 21 e 24, respectivamente). Portanto, um programa consiste em uma lista de declarações (\texttt{Statement}) que podem ser do tipo \texttt{Instr} ou \texttt{Declaration}. O tipo \texttt{Instr} contém todas as instruções básicas da linguagem ROBO, ou seja, comandos de movimentação, pintura de mapa, captura de objetos, estruturas condicionais e de repetição.
%Este novo termo permite que o compilador analise programas ROBO contendo declaração de variáveis e procedimentos. 
Já o tipo \texttt{Declaration} é define a sintaxe para declaração de variáveis e de procedimentos. 
%Assim, está disposto na Figura \ref{fig:gramatica} parte do módulo Robo2CSP, nela estão definidas as produções da linguagem.

 
A produção \texttt{Statement.Declaration} define três alternativas para uma declaração. A primeira representa variáveis (\texttt{Declaration.Variable}), a segunda procedimentos não parametrizados (\texttt{Declaration.Procedure}), e por fim, procedimentos parametrizados (\texttt{Declaration.ProcParam}). O corpo da produção para variáveis \texttt{<<Identifier> = <Expr>>} define uma declaração do tipo \texttt{Variable} composta do identificador da variável (\texttt{Identifier}), seguido por um símbolo de igual e terminado por uma expressão que pode ser do tipo booleana ou aritmética (linhas 31 e 32 da Figura \ref{fig:gramatica}, respectivamente). Esta mesma produção é usada para uma chamada de variável, o que difere uma da outra é a aplicação de regras de tradução distintas. Um procedimento não parametrizado (\texttt{<procedure <Identifier>\{ <Statement*> \}>}) é composto pela palavra reservada \texttt{procedure}, seguida do identificador do procedimento, que por sua vez é seguido por zero ou mais \texttt{Statement} dentre dos símbolos de abre e fecha chaves. Vale salientar que na linguagem ROBO um procedimento não parametrizado também pode ser escrito com abre e fecha parênteses. Dessa forma, também é possível representar um procedimento não parametrizado através da produção \texttt{Declaration.ProcParam} quando a lista de parâmetros for vazia. Um procedimento parametrizado, após o seu identificador, inclui a produção o tipo \texttt{Params} que inclui zero ou mais identificadores separados por vírgula dentro dos símbolos de abre e fecha parênteses dada pela produção \texttt{Params.Params} (linha 28).

Além da declaração de variáveis e procedimentos, a gramática foi ampliada para incluir a definição de produções para chamada de procedimento. Esta produção está indicada na linha 34 da Figura \ref{fig:gramatica}. Uma chamada de procedimento nada mais é do que um subtipo de \texttt{Instr}, chamado de \texttt{ProcCall}. O corpo desta produção é composto por um identificador seguido pelos parâmetros. A produção \texttt{ExprParams} define a lista de parâmetros de uma chamada de procedimento como expressões separadas por vírgula e inseridas dentre de parênteses.

Com a definição da gramática da linguagem ROBO em SDF3 torna-se possível a geração da árvore para representar os programas ROBO. Lembrando que as etapas para a geração da tabela de símbolos (\textit{Parse Table}) e da remoção de ambiguidades ocorre de modo implícito pelo \textit{framework}.

Usamos o programa ROBO mostrado na Figura \ref{fig:roboprogram} para ilustrar o parsing de um programa ROBO que contém variável e procedimento definido pelo usuário. A Figura \ref{fig:roboprogram} mostra um programa ROBO que resolve o problema da contagem de caixas. 
Esse programa possui duas variáveis globais: \texttt{counter} que armazena a quantidade de caixas encontradas; e \texttt{lookLeft} que indica qual a linha que o robô deve contar as caixas. O código também possui um procedimento parametrizado chamado \texttt{countBoxes} com o parâmetro chamado \texttt{side}, cujo valor indica o lado em que o robô vai contar as caixas. Se o valor de \texttt{side} é igual a 1, o robô conta as caixas no seu lado esquerdo (primeira linha do mapa);  quando o parâmetro possui outros valores o robô conta as caixas do lado direito (segunda linha do mapa). O primeiro comando do programa do robô é o comando \texttt{right} (indicado na linha 16), o qual altera sua orientação em 90 graus para a direita. O próximo passo é a execução de um laço pelo comando \texttt{repeatWhile}, o laço ocorre enquanto não houver quaisquer objetos ou paredes na célula à frente do robô. Esse laço é responsável por chamar o procedimento \texttt{countBoxes} passando o valor da variável \texttt{lookLeft}, que neste exemplo tem valor 1, ou seja, contará as caixas da linha esquerda, seguindo de um \texttt{forward} que move o robô para frente em uma unidade a cada execução do laço. Ao sair dessa estrutura de repetição, o procedimento será executado mais uma vez, com o objetivo de verificar possíveis caixas na última posição do robô e por fim a quantidade de caixas é exibida por meio do comando \texttt{show} exibindo o valor armazenado na variável \texttt{counter} contendo a quantidade de caixas na linha de interesse. 
A execução deste programa faz o robô percorrer em linha reta contando as caixas à sua esquerda e parar na posição final, quando o programa mostra o resultado 2 (quantidade de caixas encontradas na primeira linha). Caso a variável \texttt{lookLeft} seja inicializada com o valor 0, o programa contaria as caixas encontradas à direita do robô, portanto o valor impresso pelo programa seria 3. 

\begin{figure}[!h]
\caption{Programa escrito em ROBO}
\lstinputlisting[nolol]{codes/program1.rob}
\fonte{O autor}
\label{fig:roboprogram}
\end{figure}

Considerando a gramática apresentada, a Figura \ref{fig:ast1} corresponde a árvore sintática resultante do \textit{parsing} (parte dela foi omitida, versão completa disponível no Apêndice~\ref{apendice2}) do programa ROBO introduzido na Figura \ref{fig:roboprogram}. A raiz da árvore corresponde ao elemento \texttt{Program} que contém uma lista com todos os comandos (\texttt{Statement}) do programa. No programa, a primeira linha possui a declaração da variável \texttt{counter}. Na AST essa declaração é representado pelo primeiro elemento da lista (linha 2): \texttt{Declaration(Variable(ID("counter")...)}. O mesmo vale para a variável \texttt{lookLeft}. Isto é, uma declaração de variável com um identificador e uma expressão matemática. A declaração do procedimento \texttt{countBoxes} é representada na AST pelo termo \texttt{Declaration(ID("countBoxes), ...)} (linhas 4 a 8). O comando \texttt{right} na árvore é dado por \texttt{Instr(RIGHT())}. O comando para executar um laço (\texttt{repeatWhile}) é equivalente a \texttt{Instr(RPTWLE(...))} (linhas 11 a 20), onde o primeiro elemento de \texttt{RPTWLE} é \texttt{FROISCLR} para representar o comando \texttt{frontIsClear}, e o segundo é uma representação de \texttt{forward} (\texttt{Instr(FORWARD(...)}). Em uma chamada do procedimento \texttt{countBoxes}, sua representação na árvore é denotada por \texttt{Instr(ProcCall(...))} (linhas 21 a 23), ou seja, um identificador, e uma lista de parâmetros no termo \texttt{ExprParams}. Por fim, temos o comando \texttt{show} que é representado por \texttt{Instr(SHOW(...))}.

\begin{figure}[!h]
\centering
\caption{AST de um programa ROBO}
\lstinputlisting[nolol]{codes/ast2.aterm}
\fonte{O autor}
\label{fig:ast1}
\end{figure}

A análise semântica de ROBO até então não realizada pelo compilador. Embora espera-se que o usuário escreva os programas no ambiente RoboMind, o que mostraria erros semânticos, pode haver casos de programas escritos fora desse ambiente. A atual abordagem deixa a cargo do FDR encontrar problemas semânticos da especificação CSP. Portanto, o usuário precisa ter minimamente um conhecimento da notação formal para entender o problema e assim resolver em ROBO. Sendo assim, sendo uma das limitações da nossa abordagem.
%Mas só com definição da gramática ainda não é possível gerar a especificação formal do programa, neste ponto é preciso realizar transformações na árvore para alcançar tal objetivo. Desse modo, é introduzido na próxima seção a geração de código CSP através da linguagem Stratego.

As alterações na gramática descritas acima permitem representar sintaticamente programas escritos em ROBO com variáveis e procedimentos e gerar sua árvore sintática. Esta árvore é a entrada para obter a representação formal para o programa em CSP que é apresentada na próxima seção.

\section{Transformação com Stratego}

%Essa etapa é uma das mais importantes, em razão de que o resultado é a geração de código CSP, um produto essencial para a verificação dos programas ROBO no verificador de modelos FDR.

%Como dito anteriormente, a análise automática de programas ROBO só é possível devido a semântica CSP bem definida para cada elemento sintático da linguagem ROBO. Esta seção mostra as regras de transformação utilizadas para definir os elementos de CSP que representam variáveis e procedimentos.

Conforme mostrado na seção anterior, a gramática SDF3 foi estendida para aceitar variáveis e procedimentos definidos pelo usuário. Esta seção introduz regras de tradução que mapeiam os elementos da AST que representam variáveis e procedimentos para os respectivos elementos em CSP. 

A Seção~\ref{sec:csp} mostrou o modelo CSP que representa um programa robô armazena a posição e orientação do robô em um processo que faz o papel de memória. A seguir, mostramos como Stratego foi utilizado para coletar na AST as variáveis definidas pelo usuário e incluí-las no processo que representa a memória. Além das variáveis, Stratego foi utilizado para coletar da AST do programa os procedimentos definidos pelo usuário e produzir a representação dos procedimentos em CSP.

Para facilitar o entendimento das regras propostas, a Figura~\ref{fig:progcsp} mostra parte da especificação formal gerada automaticamente, como resultado da aplicação das regras que serão introduzidas a seguir, quando a entrada das regras é a AST mostrada na Figura~\ref{fig:ast1}. No Apêndice~\ref{cod:progadap} está disposta a especificação formal completa dessa tradução.
%Em vista das explicações mencionadas à respeito das regras de tradução implementadas para procedimentos e variáveis, entende-se que de agora em diante é possível gerar notação CSP automaticamente para programas ROBO. 
Na Figura~\ref{fig:progcsp} podemos notar que as variáveis \texttt{counter} e \texttt{lookLeft} são representadas pela constantes \texttt{counterConst} e \texttt{lookLeftConst} na especificação CSP. Também foi gerado o tipo de dados para cada variável (linha 5). Além de gerar o conjunto \texttt{INIT} para a inicialização do processo \texttt{MEMORY} com os valores iniciais de cada variável (linha 6). O procedimento \texttt{countBoxes} virou o processo \texttt{countBoxesProc} e o seu parâmetro \texttt{side} foi chamado de \texttt{sideParam}, isso ocorreu porque quando um processo parametrizado for chamado o valor do parâmetro é armazenado na memória (\texttt{set.side!(sideParam)}), possibilitando sua atualização como uma variável local do procedimento. Na linha 36, mostra uma chamada do processo \texttt{countBoxesProc} com o parâmetro \texttt{lookLeftVar} que foi obtido na linha anterior por meio do evento \texttt{get.lookLeft?lookLeftVar}, que busca o valor da variável na memória. O mesmo vale para o evento \texttt{showInt}, que é equivalente ao comando \texttt{show} em ROBO. Para valores booleanos a instrução \texttt{show} corresponde ao evento \texttt{showBool} em CSP. Esta figura omite a definição das células da memória e outros processos que foram apresentados na Seção~\ref{sec:csp} e fazem parte do modelo CSP para um program ROBO.  

\begin{figure}[!h]
\centering
\caption{Especificação CSP gerada a partir de um programa ROBO}
\lstinputlisting[nolol]{codes/program.csp}
\fonte{O autor}
\label{fig:progcsp}
\end{figure}

A Figura~\ref{fig:rules} mostra a regra \texttt{main-to-csp} (linha 3) que transforma \texttt{Program(T1)}, que corresponde a raiz da AST, em uma especificação CSP. O termo \texttt{T1} corresponde ao restante da árvore sintática do programa. Esta regra é a primeira regra a ser executada por Spoofax durante o processo de tradução. O CSP gerado possui uma estrutura fixa como a definição de constantes (linha 5), o domínio das variáveis definidas pelo usuário (linha 6), a definição de variáveis (linha 7), a inicialização da memória (linha 8), procedimentos (linha 10) e os comandos do programa (linha 12). Dentro desta estrutura fixa existem elementos que aparecem entre parêntesis e são substituídas por fragmentos de especificação CSP obtidos pela aplicação de outras regras, que usam como entrada o termo \texttt{T1}. Por exemplo, o elemento \texttt{[proc']} (linha 10) corresponde ao modelo CSP para os procedimentos definidos pelo usuário. Os elementos entre parêntesis são definidos nas linhas 16 a 22 da Figura~\ref{fig:rules}. Explicamos cada um destes elementos.

%O primeiro passo foi definir uma regra em Stratego para manipular os principais termos de um AST. No qual, para cada termo, outras regras são aplicadas, isto é, o conjunto resultante de termos aplicados à uma regra é utilizado pela regra consecutiva. Isto é melhor exemplificado na regra definida na Figura \ref{fig:rules}. A regra \texttt{main-to-csp} é responsável por aplicar um cabeçalho para os programas em CSP e desencadear outras regras para os demais termos da árvore. Como a AST de ROBO sempre inicia com \texttt{Program}, isso siginifica que toda geração de código é iniciada por essa regra, no caso ocorre o casamento com o termo \texttt{Program(T1)}, indicado na linha 4, onde \texttt{T1} é todo o restante da árvore sintática. Diante disso, todas as demais regras são aplicadas aos termos derivados de \texttt{T1}.

\begin{figure}[!h]
\centering
\caption{Regra inicial para um programa ROBO}
\lstinputlisting[nolol]{codes/rules.str}
\fonte{O autor}
\label{fig:rules}
\end{figure}

%Como visto na Figura \ref{fig:rules}, está definida a regra \texttt{main-to-csp} na qual o conteúdo que está entre as linhas 5 e 13 será transformado em notação CSP. A linha 5 contém o elemento \texttt{vars'} que é utilizado para transformar todas as variáveis de ROBO em constantes em CSP. Esse elemento é transformado após a palavra \texttt{with}, no qual outras duas regras são aplicadas em \texttt{T1} e o valor resultante é aplicado a \texttt{vars'}. 

O elemento \texttt{vars'} é obtido pela aplicação da regra \texttt{<get-vars>} em \texttt{T1}, essa regra tem como objetivo obter uma lista, através da aplicação de filtro na árvore, com o conjunto de \textit{ATerms} que casam com o termo \texttt{Declaration(Variable(\_,\_))}. A definição desta regra está na linha 6 da Figura \ref{fig:rules2}. Após isso, uma lista com os termos que correspondem a declaração de variáveis são a entrada da regra mais a esquerda, \texttt{<var-analyze-const>}, que gera como saída  as variáveis declaradas no programa ROBO em forma de constantes CSP, a Figura~\ref{fig:progcsp} ilustra nas linhas 1 e 2 as variáveis \texttt{counter} e \texttt{lookLeft} como constantes. A Figura \ref{fig:rules_constants} mostra a definição das regras responsáveis pela geração das constantes. A regra \texttt{<var-analyze-const>} é aplicada de modo recursivo, muito similar às funções de linguagens funcionais, no qual uma função é aplicada na cabeça (\textit{head}) enquanto a mesma função é aplicada na calda (\textit{tail}) até chegar no caso base, que neste é caso é uma lista vazia, como está indicado na linha 3 da Figura~\ref{fig:rules_constants}. A análise ocorre em cima do termo \texttt{Declaration(var)}, onde a regra \texttt{<write-variable-const>} é aplicada em \texttt{var} para escrever o CSP correspondente ao tipo da variável. Por isso, há três declarações dessa regra, a primeira para escrever um texto vazio em casos de expressões matemáticas, já que as constantes não recebem expressões (exemplo, a + b); a segunda para escrever o texto \texttt{[name]Const = [v]}, ou seja, o nome da variável (\texttt{name}) e seu respectivo valor inteiro (\texttt{v}); por último, para expressões booleanas, onde uma outra regra, \texttt{<to-csp-e>}, é aplicada para escrever o valor booleano (\textit{true} ou \textit{false}), dependendo do valor do termo em \texttt{exp'}. A regra \texttt{<to-csp-e>} está definida no trabalho~\cite{nogueira}. A transformação das variáveis de ROBO em constantes no modelo CSP é importante para os casos em que em uma declaração de variável que haja outra variável na expressão (exemplo, \texttt{a = b + 1}). Em CSP não é possível utilizar o nome da variável diretamente à \texttt{INIT}, pois é necessário que um evento \texttt{get} tenha sido comunicado com o valor da variável utilizada na expressão da declaração, o que é realizado apenas dentro de processos. As variáveis como constantes resolvem este problema, uma vez que pode ser utilizada por \texttt{[n]Const}, onde \texttt{n} é o identificador da variável. Em \texttt{INIT} seriam adicionados da seguinte forma: \texttt{(a, (bConst + 1))}.

\begin{figure}[!h]
\centering
\caption{Conjunto de regras auxiliares}
\lstinputlisting[nolol]{codes/rulesAux.str}
\fonte{O autor}
\label{fig:rules2}
\end{figure}

\begin{figure}[!h]
\centering
\caption{Regras para geração de constantes em CSP}
\lstinputlisting[nolol]{codes/rules_constants.str}
\fonte{O autor}
\label{fig:rules_constants}
\end{figure}

O conjunto \texttt{INTVALUES} na linha 6 da Figura \ref{fig:rules} define os valores que as variáveis inteiras definidas pelo usuário podem antigir durante a execução do programa. Este conjunto é necessário pois em FDR quando o conjunto numérico é muito grande, a verificação poder ter o desempenho comprometido devido a grande quantidade de estados que o programa possui. A definição deste conjunto deve ser adaptada manualmente para os valores sejam adequados para os programas que serão analisados. Isto é uma das limitações atuais da abordagem.

Foi explicado no Capítulo~\ref{chap:cap2}, que em CSP existe um processo (\texttt{MEMORY}) que armazena informações de estado do robô e de objetos no mapa. Usamos este mesmo processo para armazenar as variáveis definidas pelo usuário, em adição às variáveis da posição e a orientação do robô. Na linha 7 da Figura~\ref{fig:rules}, o tipo das variáveis é definido por \texttt{VarType}, que corresponde a junção dos tipos dos parâmetros (\texttt{paramsType'}) e das variáveis (\texttt{varsType'}) encontrados nos programas ROBO. Uma limitação disto é que variáveis globais e locais (aos procedimentos) são colocadas no mesmo escopo. Desde que as variáveis tenham nomes diferentes, não acontece conflito de nomes na memória.   

%Para gerar a especificação formal em relação ao tipo de dados para parâmetros e variáveis através dos elementos \texttt{paramsType'} e \texttt{varsType'} é necessário aplicar algumas regras para recolher os termos de interesse na árvore sintática. 
Para a obtenção de \texttt{paramsType'}, a primeira regra aplicada é \texttt{<get-params>} (linha 8 da Figura \ref{fig:rules2}). Esta regra usa a função nativa do Stratego chamada de \texttt{collect-all}, cujo objetivo é recolher todos os termos que são alcançados a partir da raiz de uma AST. Portanto, o retorno desta regra é uma lista contendo todos os termos \texttt{Params(\_)} do programa ROBO. Os parâmetros são a entrada para a regra \texttt{<get-ids>}, que recolhe todos os identificadores (\texttt{ID(\_)}) dos parâmetros. Os identificadores são a entrada para a regra \texttt{<param-analyze-type>}, que está detalhada na Figura \ref{fig:rules_param_type}. Esta última regra gera os tipos dos parâmetros no formato de CSP. Essa regra é aplicada também de modo recursivo, a recursão ocorre na linha 4 na Figura \ref{fig:rules_param_type}, onde o primeiro elemento da lista \texttt{ID(n)} é analisado e destinado à regra \texttt{<write-param-type>} que é aplicada em \texttt{n}, enquanto para o restante da lista (\texttt{es}) a regra principal é aplicada recursivamente. A regra \texttt{write-param-type} escreve o CSP, o conteúdo que está entre os símbolos cifrão e colchetes é convertido em texto e \texttt{[n]} é substituído pelo nome do parâmetro analisado. Para exemplificar, o termo \texttt{ID("side")}, visto na linha 6 da Figura \ref{fig:ast1}, é escrito em CSP como \texttt{side.INTVALUES}. Uma limitação da tradução atual é que ela funciona apenas para parâmetros do tipo inteiro.
%após todo esse processo descrito, o mesmo vale para todos os parâmetros em qualquer código ROBO.

\begin{figure}[!h]
\centering
\caption{Regras para os tipos de dados de parâmetros em notação CSP}
\lstinputlisting[nolol]{codes/rules_param_type.str}
\fonte{O autor}
\label{fig:rules_param_type}
\end{figure}

%De modo semelhante ao que é feito em \texttt{paramsType'}, também ocorre em 
As regras usadas para obter \texttt{varsType'} são semelhantes àquelas usadas para obter \texttt{paramsType'}. A diferença está em como ocorre a escrita de CSP, uma vez que as variáveis ROBO possuem dois tipos de dados em suas expressões: \texttt{Bool}, para expressões com valores booleanos e \texttt{INTVALUES}, para expressões com valores inteiros. A regra \texttt{<write-variable-type>}, como mostra nas linhas 6 e 9 da Figura~\ref{fig:rules_var_type}, aparece duas vezes, a primeira para termos que possuem expressões aritméticas e a segunda para os termos com expressões booleanas. Por exemplo, na linha 5 da Figura~\ref{fig:progcsp} é ilustrado a adição dos tipos das variáveis \texttt{side}, \texttt{counter} e \texttt{lookLeft}.

\begin{figure}[!h]
\centering
\caption{Regras para os tipos de dados de variáveis em notação CSP}
\lstinputlisting[nolol]{codes/rules_var_type.str}
\fonte{O autor}
\label{fig:rules_var_type}
\end{figure}

%Foi explicado no Capítulo \ref{chap:cap2}, que em CSP é necessário um processo para simular uma memória que armazena os valores das variáveis e demais dados do robô. Assim, a escrita do CSP de parâmetros e variáveis em \texttt{INIT}, através dos elementos \texttt{paramsInit'} e \texttt{varsInit'}, ocorre de modo análogo para os tipos de dados de parâmetros e variáveis aos valores das posição do robô e orientação.

Os elementos \texttt{paramsInit'} e \texttt{varsInit'} que aparecem na Figura~\ref{fig:rules} (linha 8) correspondem ao valor inicial dos parâmetros de procedimento e variáveis definidas pelo usuário. Estes valores são colocados na constante \texttt{INIT} que é utilizada para definir o estado inicial da memória. A transformação ocorre de modo semelhante as regras para os tipos de variáveis e parâmetros. Na Figura~\ref{fig:rules_proc_init} está definida a regra \texttt{<param-analyze-init>} para adicionar a essa constante os parâmetros dos procedimentos. Essa regra percorre recursivamente a lista de termos aplicando a regra \texttt{<write-param-init>} no termo analisado. Ou seja, todos os parâmetros são definidos em CSP como \texttt{([n], 0)}, onde \texttt{n} é o identificador do parâmetro. Essa abordagem foi proposta para garantir que um parâmetro possa ser atualizado durante a execução de um procedimento. Por exemplo, na Figura~\ref{fig:progcsp} mostra o parêmatro \textit{side} como uma variável à \texttt{INIT} (linha 6). A limitação é que apenas valores inteiros são considerados. Na Figura~\ref{fig:rules_var_init} estão dispostas as regras para inicialização das variáveis. A regra \texttt{<var-analyze-init>} é aplicada recursivamente nos termos, onde a regra \texttt{<write-variable-init>} é aplicada ao valor de \texttt{var} do termo \texttt{Declaration(var)}. Essa regra gera o CSP para as variáveis no formato \texttt{([name].[exp'])}, onde \texttt{name} é o identificador da variável e \texttt{exp'} a expressão que pode ser booleana ou aritmética, por isso há duas definições de \texttt{<write-variable-init>}. Na Figura~\ref{fig:progcsp}, a aplicação dessa regra está ilustrada na linha 6 (variáveis \texttt{counter} e \texttt{lookLeft}).

\begin{figure}[!h]
\centering
\caption{Regras para os tipos de dados de variáveis em notação CSP}
\lstinputlisting[nolol]{codes/proc_init.str}
\fonte{O autor}
\label{fig:rules_proc_init}
\end{figure}

\begin{figure}[!h]
\centering
\caption{Regras para os tipos de dados de variáveis em notação CSP}
\lstinputlisting[nolol]{codes/var_init.str}
\fonte{O autor}
\label{fig:rules_var_init}
\end{figure}

O elemento \texttt{proc'} que aparece na linha 10 da Figura \ref{fig:rules} corresponde a especificação CSP dos procedimentos.  
Como existem dois tipos de procedimentos na linguagem ROBO, os parametrizados e não parametrizados, é necessário aplicar individualmente as regras \texttt{get-procs} e \texttt{get-procs-param} (definidas nas linhas 3 e 4 da Figura \ref{fig:rules2}) em \texttt{T1} para recolher todos os termos relacionados aos procedimentos e depois combiná-los em uma única lista.
%Na regra principal, \texttt{<main-to-csp>}, Figura \ref{fig:rules}, está definido o elemento \texttt{proc'} na linha 10, ele é responsável expressar todos os procedimentos dos programas ROBO em um formato compatível com os processos CSP, mas que sejam semanticamente iguais ao procedimento escrito na linguagem ROBO. 
Na linha 21 da Figura \ref{fig:rules} é utilizada a regra \texttt{<union>}  nativa do \textit{framework}, cujo objetivo é unir duas listas de termos.  A lista resultante é a entrada da regra \texttt{<statement-definition-decl>}, exposta na Figura \ref{fig:rules_proc} que produz o CSP dos procedimentos.

\begin{figure}[!h]
\centering
\caption{Regras para a geração de CSP dos procedimentos}
\lstinputlisting[nolol]{codes/rules_proc.str}
\fonte{O autor}
\label{fig:rules_proc}
\end{figure}

A regra \texttt{<statement-definition-decl>}, definida nas linhas 3 e 4 da Figura \ref{fig:rules_proc}, aplica a regra \texttt{<to-csp>} para cada declaração (\texttt{Declaration(s)}) encontrada na lista recebida como entrada. 
%A escrita de notação CSP ocorre na regra \texttt{<to-csp>} que é aplicada ao conteúdo de \texttt{s} no termo \texttt{Declaration(s)}. 
Quando a declaração casa com o padrão \texttt{ProcParam(ID(name), Params(params), procedureBody} é executada a regra na linha 6 da Figura~\ref{fig:rules_proc} que gera a especificação CSP para o procedimento. Sendo \texttt{name} o nome do procedimento; \texttt{params} a lista de parâmetros; e \texttt{procedureBody} as instruções dentro do corpo do procedimento. Na especificação CSP, o nome do processo que representa o procedimento é definido como o nome do procedimento em ROBO concatenado com a palavra \texttt{Proc}, a Figura~\ref{fig:progcsp} ilustra através do procedimento \texttt{coutBoxesProc} (linha 8). O nome do processo é seguido pelos parâmetros (\texttt{params'}), o símbolo de igual, o elemento \texttt{paramVar'} e \texttt{procedureBody'}. Como os parâmetros são colocados na memória, o CSP do elemento \texttt{paramVar'} corresponde à atualização do valor do parâmetro na memória usando o valor recebido como parâmetro usando o canal \texttt{set}. O CSP deste elemento é obtido aplicando a regra \texttt{<put-param-mem>} definida na Figura \ref{fig:put_proc}. A representação de um parâmetro como uma variável permite que o parâmetro seja utilizado com uma variável local dentro do procedimento, que pode ser tanto lida como atualizada pelo programa. 
%objetivo de resolver o problema de atualização de valor dos parâmetros, pois em ROBO é possível atualizar o valor do parâmetro a qualquer momento dentro um procedimento. Em tal caso, foi proposta uma abordagem onde cada parâmetro de um procedimento é adicionado à memória antes das instruções de um mesmo procedimento. 
%A Figura \ref{fig:put_proc} destaca as regras responsáveis por isso, a especificação CSP é gerada na linha 4. 
Vemos na Figura \ref{fig:roboprogram} linha 9 um exemplo de aplicação desta regra. O parâmetro \texttt{side} no procedimento \texttt{countBoxes} é representado no  CSP pelo parâmetro \texttt{sideParam}, cujo valor é usado para atualizar a variável \texttt{side} na memória através do evento \texttt{set.side!(sideParam)}.

\begin{figure}[!h]
\centering
\caption{Regras que adicionam os valores dos parâmetros na memória em uma chamada de procedimento}
\lstinputlisting[nolol]{codes/put-var-proc.str}
\fonte{O autor}
\label{fig:put_proc}
\end{figure}

Depois de obter o CSP de \texttt{paramVar'}, ocorre a geração de código para o corpo do procedimento.
%, onde um procedimento poder conter diferentes instruções, seja para movimentar robô ou chamar inclusive outro procedimento, além de todas as estruturas condicionais e de repetição. Ainda 
Na Figura \ref{fig:rules_proc}, o elemento \texttt{procedureBody'} é obtido aplicando a regra \texttt{<statement-definition>} que é explicitada na Figura \ref{fig:statement}. Essa regra aplica recursivamente a regra \texttt{<to-csp>} para todos os termos que representam comandos dentro do corpo do procedimento; quando não existem mais comandos para traduzir é escrito o processo \texttt{SKIP}, que é um processo CSP que termina com sucesso.

\begin{figure}[!h]
\centering
\caption{Regras que aplica \texttt{to-csp} para cada instrução}
\lstinputlisting[nolol]{codes/rules_statement.str}
\fonte{O autor}
\label{fig:statement}
\end{figure}

A regra \texttt{<to-csp>} possui múltiplos padrões introduzidos em~\cite{nogueira}. Para cada termo que representa uma construção da linguagem ROBO existe uma regra \texttt{<to-csp>} que gera o CSP correspondente.
%Isso ocorre porque existem diferentes tipos de instruções, algumas delas estão representadas na árvore mostrada na Figura \ref{fig:ast1}, como por exemplo,
Adicionamos mais dois padrões para a regra \texttt{<to-csp>} para gerar o CSP que corresponde aos termos \texttt{ProcCall} e \texttt{SHOW} da gramática. Na Figura \ref{fig:to_csp} são mostrados estes padrões. O primeiro (linha 3) gera o CSP para a chamada do procedimento. O segundo (linha 13) gera o CSP do comando que exibe valores inteiros no \textit{console}.
%, que neste caso gera a notação CSP para exibir valores inteiros. 
Por concisão, omitimos o padrão que gera CSP para o comando que exibe valores booleanos. 

Detalhamos a regra que gera o CSP para uma chamada de procedimento. O CSP gerado para uma chamada de procedimento 
%Nesta regra foi proposto uma abordagem para antes de uma chamada de procedimento, em CSP, deve-se 
obtém os valores das variáveis que são passadas como parâmetro para a chamada, conforme mostrado na linha 5 da Figura~\ref{fig:to_csp}, que é representado pelo elemento \texttt{vars'}. Para obter o CSP deste elemento é aplicada a regra \texttt{<get-vars-exp>} que tem o objetivo de recolher todas as variáveis presentes na expressão (\texttt{ExprParams()}). Em seguida, é aplicada a regra \texttt{<put-get-var-exp-analyze>}, cujo objetivo é colocar todas as variáveis no formato de CSP, como mostra na Figura \ref{fig:rules_var}. Na linha 6 ocorre a escrita do texto CSP pela regra \texttt{<put-get-var-exp>}, por exemplo, quando a variável \texttt{lookLeft} em ROBO é passada como parâmetro para uma chamada de procedimento sua leitura é escrita como a comunicação do evento \texttt{get.lookLeft?lookLeftVar} no modelo formal à linha anterior a chamada do processo \texttt{countBoxesProc(lookLeftVar)} (linhas 36 e 37 da Figura~\ref{fig:progcsp}). Já para escrever o nome dos parâmetros na chamada do procedimento é aplicada a regra  \texttt{<get-param-names-call>} nos termos de \texttt{params}. Lembrando que em uma chamada de procedimento é possível passar qualquer expressão inteira como parâmetro, portanto a regra está preparada para isto. A Figura \ref{fig:rules_param} mostra as regras auxiliares para obter o CSP da chamada do procedimento.
%, uma vez que há expressões matemáticas e booleanas, além do mais a aplicação dela ocorre de modo recursivo. Por exemplo, na linha 7, mostra o momento em que é empregada a regra \texttt{<to-csp-e>} para o elemento \texttt{exp'} que tem o propósito de gerar CSP das expressões. O mesmo argumento é aplicável para as demais derivações dessa regra.

\begin{figure}[!h]
\centering
\caption{Exemplos da regra \texttt{to-csp}}
\lstinputlisting[nolol]{codes/tocsp.str}
\fonte{O autor}
\label{fig:to_csp}
\end{figure}

\begin{figure}[!h]
\centering
\caption{Regras para gerar parâmetros em uma chamada de procedimento}
\lstinputlisting[nolol]{codes/rules_proc_call.str}
\fonte{O autor}
\label{fig:rules_param}
\end{figure}


Outra instância de \texttt{<to-csp>} foi definida para atualização de variáveis, indicada pela linha 10 na Figura~\ref{fig:rules_var}. Na linha 12 é possível notar quando o CSP é gerado com intuito de escrever uma consulta do valor da própria variável, enquanto as demais variáveis são representadas pelo elemento \texttt{var'}, na linha 13. Na Figura~\ref{fig:progcsp}, linhas 18 e 19, mostra um exemplo com a variável \texttt{counter}, onde ela é utilizada na expressão \texttt{counter + 1}, dessa forma uma consulta é realizada (\texttt{get.counter?counterVar}) antes da atribuição. Essa abordagem é necessária porque uma variável pode ser atualizada por ela mesma, além de uma combinação de expressões com diferentes variáveis e valores. O CSP da atualização do valor dela na memória ocorre na linha 14. Para fins práticos, considerando a expressão \texttt{counter = counter + 1}, ela seria escrita em CSP como \texttt{member((countBoxesVar + 1), INTVALUES) \& set.countBoxes!((countBoxesVar + 1))}. Isso quer dizer que primeiro é verificado se o novo valor da variável está dentro do limite inferior e superior para que de fato seja atualizada na comunicação do evento \texttt{set}. A verificação que ocorre com a função \texttt{member} é para garantir que o novo valor da variável pertence ao conjunto \texttt{INTVALUES}, caso contrário o evento \texttt{set} não é comunicado, e consequentemente a variável não será atualizada, tal fato é uma limitação dessa abordagem.

\begin{figure}[!h]
\centering
\caption{Regras para buscar e atualizar valores das variáveis}
\lstinputlisting[nolol]{codes/rules_var.str}
\fonte{O autor}
\label{fig:rules_var}
\end{figure}


A parte final da tradução corresponde ao elemento \texttt{instr'} (Figura~\ref{fig:rules} linha 13), que representa o CSP dos comandos do programa que aparecem fora dos procedimentos, isto é, os eventos que são adicionados dentro do processo \texttt{COMMANDS}.
%Todas as instruções de um programa ROBO são adicionadas dentro do processo chamado de \texttt{COMMANDS} em CSP onde todas instruções são representadas pelo elemento \texttt{instr'} indicado na linha 22 da Figura \ref{fig:rules}.
%Como o foco deste trabalho são as regras para procedimentos e variáveis, serão explicadas apenas as definições de \texttt{<to-csp>} que representam instruções dessa natureza. A primeira definição de \texttt{<to-csp>} é para chamada de procedimentos que está definida na Figura \ref{fig:to_csp}.

A tradução é contemplada com as regras de compilação iniciais apresentadas em \cite{nogueira} para as instruções básicas do robô, agora também inclui uma extensão do compilador com a adição de regras para programas que possuem declarações de procedimentos e variáveis e chamadas de procedimentos e atualizações de variáveis, além de outras melhorias.


\chapter{Implementação}

Este capítulo tem por objetivo mostrar, em um exemplo prático, resultados de uso do compilador desenvolvido nesta pesquisa em conjunto as demais ferramentas de modo que o processo de verificação ocorra como esperado. Na seção \ref{sub:sec41} é explicado como foi realizada a integração do compilador com o verificador de modelos. Na seção x 

\section{Ferramenta de Verificação (RobotChecker)*}
\label{sub:sec41}

Um dos objetivos desta pesquisa é o desenvolvimento de um protótipo funcional de uma ferramenta que integre o compilador com o verificador de modelos de modo transparente ao usuário do RoboMind. Posto isso, foi proposto um protótipo desenvolvido em Java de um serviço que provê as principais funcionalidades para a verificação automática dos programas ROBO. Para simular em um ambiente real, foi desenvolvido um método que é responsável por enviar os programas ROBO e seus respectivos mapas ao serviço, e como resultado é obtido o resultado da verificação e um contra-exemplo (\textit{traces}), caso existir.

No protótipo proposto foram implementados os principais métodos para realizar a integração do tradutor com o FDR:
\begin{itemize}
    \item Tradução de mapa: recebe um mapa do ambiente RoboMind como entrada e realiza a tradução automática; e como saída é gerado a especificação formal em CSP do mapa.
    \item Tradução de ROBO para CSP: recebe um programa ROBO como entrada e chama a API do Spoofax para realizar a tradução usando o compilador desenvolvido por esta pesquisa e como saída é gerado um CSP equivalente.
    \item Verificação das propriedades no FDR: recebe como entrada o mapa e o programa escritos em CSP, resultado da tradução automatizada; e como saída gera os resultados das verificações e os \textit{traces} resultantes.
\end{itemize}

Na seção seguinte estão definidos os passos executados para validar a integração do compilador com o FDR.

\section{Experimento Realizado}
\label{sub:sec42}
No Capítulo \ref{chap:cap3} foi apresentado um exemplo adaptado de \cite{furb} que foi utilizado para mostrar na prática o desenvolvimento do compilador. Para este experimento, vamos utilizar esse mesmo exemplo, no entanto, em sua versão completa que é descrita na subseção \ref{sub:subdefprob}.

\subsection{Definição do problema}
\label{sub:subdefprob}
O problema consiste em um mundo de 6 colunas e 3 linhas, onde o robô sempre inicia na primeira coluna da segunda linha e e com caixas distribuídas aleatoriamente na primeira e última linha. O objetivo é fazer o robô andar até a última coluna e dizer a quantidade de ciaxas que existem na prieira e última linha, e dizer quais das duas linhas tem mais caixas: sendo 1 para a primeira linha, 2 para a última e 3 se ambas possuem a mesma quantidade. Outro ponto, é que se a quantidade não for igual, deve-se dizer a diferença de caixas que existem entre as linhas. O robô também deve dizer em qual das linhas apareceu a primeira caixa e em qual das linhas apareceu a última caixa: 1 - para a primeira linha; 2 - para a segunda; e 3 para ambas. Para melhor ilustar o experimento, considere os mapas apresentados na Figura \ref{fig:problem}, para cada um dos três mapas apresentados o robô deve mostrar as respectivas saídas.

\begin{figure}[h]
\centering
\caption{Mapas e saídas esperadas do problema ``Contando Caixas"}
\includegraphics[height=10cm]{figuras/problema.png}
\fonte{\cite{furb}}
\label{fig:problem}
\end{figure}

Definido o problema, como próximo passo foi desenvolver uma solução para esse problema. Então desenvolvemos um programa ROBO capaz de apresentar as mesmas saídas para cada um desses mapas. A solução que propusemos contém cinco procedimentos e algumas variáveis que são suficientes para resolver o problema em questão. Na Figura \ref{fig:solution} tem um trecho do código ROBO proposto. Podemos notar que há uma estrutura de repetição (\texttt{repeatWhile}) com uma condição (\texttt{frontIsClear}) que verifica se a célula posterior ao robô está livre. Os cinco procedimentos definidos são: (1) \texttt{countLeft}, conta a quantidade de caixas à esquerda do robô; (2) \texttt{countRight}, conta a quantidade de caixas à direita do robô; (3) \texttt{showsMoreBoxes}, mostra qual linha tem mais caixas e a diferença entre elas; (4) \texttt{getBoxsFirstLine}, mostra a quantidade de caixas na primeira linha; por fim, (5) \texttt{getBoxLastLine}, mostra a quantidade de caixas na última linha.

\begin{figure}[h]
\centering
\caption{Solução proposta para o problema Contando Caixas}
\lstinputlisting[language=Java]{codes/solution.rob}
\fonte{O autor}
\label{fig:solution}
\end{figure}

Para validar a efetividade do compilador em integração com FDR, foi necessário verificar as saídas geradas pelo prótotipo e executar os traces gerados no ambiente RoboMind e as saídas geradas pelo FDR, além disso o FDR provê se a solução está livre de \textit{deadlock}, uma propriedade que o RoboMind não verifica. Na próxima seção estão os resultados obtido em ambas ferramentas.

\subsection{Resultados gerados}
\label{sub:sec43}

\begin{table}[]
\caption{Resultado obtido após a verificação do FDR}
\resizebox{\textwidth}{!}{%
\begin{tabular}{*{14}{|c}|}
\multicolumn{1}{c}{Map 1} & \multicolumn{1}{c}{Map 2} & \multicolumn{1}{c}{Map 3} \\
assert PROGRAM :{[}deadlock free {[}F{]}{]} & assert PROGRAM :{[}deadlock free {[}F{]}{]} & assert PROGRAM :{[}deadlock free {[}F{]}{]} \\
Result: Failed & Result: Failed & Result: Failed \\
Counterexample - Traces: & Counterexample - Traces: & Counterexample - Traces: \\
right() & right() & right() \\
forward(1) & forward(1) & forward(1) \\
forward(1) & forward(1) & forward(1) \\
forward(1) & forward(1) & forward(1) \\
forward(1) & forward(1) & forward(1) \\
forward(1) & forward(1) & forward(1) \\
show(2) & show(2) & show(2) \\
show(3) & show(0) & show(2) \\
show(2) & show(1) & show(3) \\
show(1) & show(2) & show(0) \\
show(3) & show(1) & show(2) \\
show(3) & show(1) & show(1)
\end{tabular}%
}
\fonte{O autor}
\end{table}

\subsection{Análise dos resultados}
\label{sub:sec44}
Levando-se em conta o que foi observado, para os três mapas analisados, comparando as saídas geradas pelo RoboMind e pela ferramenta proposta, é notório que a tradução dos programas foram realizadas corretamente, isto é, os programas CSP são semanticamente equivalentes a ROBO.



\chapter{Conclusão}
\section{Trabalhos Relacionados}

Até o momento nas pesquisas levantadas, trabalhos que realizem verificação de programas de robôs educacionais de forma totalmente automática além da abordagem introduzida em \cite{nogueira} não foram encontrados. Os trabalhos da literatura que usam verificação automática têm o foco na resolução de outros problemas. O trabalho mais relacionado com o este projeto é \cite{SVA}, que propõe SVA (Shared Variable Programming), uma ferramenta para o aprendizado de programação concorrente usando uma linguagem educacional; integra um compilador da linguagem educacional para CSP e o verificador de modelos FDR. Esta ferramenta permite analisar propriedades dos programas relacionadas a concorrência como, por exemplo, se dois programas entram ao mesmo tempo em uma região crítica. Além disso, possui uma interface gráfica que apresenta os resultados das verificações retornados por FDR. Esse trabalho mostra que é possível utilizar o FDR e CSP para a criação de ferramentas de verificação automática de linguagens de domínio específico. 

Em \cite{nogueira} é proposto uma abordagem de verificação automática dos programas ROBO por meio do verificador de modelos FDR. Inicialmente são fornecidos o programa e o mapa para serem traduzidos para CSP, posteriormente são definidas quais propriedades serão verificadas, e por fim é realizado a verificação por meio da ferramenta FDR. Uma limitação desta abordagem é que o mecanismo automático de tradução dos programas não está completamente desenvolvido, pois não contempla a tradução de todos os elementos da linguagem ROBO para CSP. Outra limitação, é que não existe uma interface para integrar o processo de tradução com a verificação de propriedades realizadas por FDR.

Em Oliveira et al. (2017), é proposto um protótipo de um ambiente para avaliação automática de robôs virtuais. Esse trabalho é uma continuação do trabalho apresentado no parágrafo anterior com foco no projeto e prototipação da interface gráfica de um ambiente para avaliação automática de robôs virtuais. O ambiente proposto objetiva a avaliação de forma automática e feedback sobre o funcionamento dos programas escritos em ROBO, através de uma interface gráfica que lembra um sistema de julgamento online (Online Judgment System) utilizados nas maratonas de programação. A implementação deste ambiente depende diretamente de um dos produtos deste trabalho, que é a extensão do compilador de ROBO para CSP e sua integração com a ferramenta FDR.

O trabalho \cite{silva} apresenta um método para a verificação automática durante a simulação de futebol de robôs, esse trabalho considera a especificação formal e a verificação de planos de um time de robôs simulados. A simulação ocorre de modo que vários robôs estão executando ao mesmo tempo e propondo uma solução conjunta, onde o autor denomina como sistemas multiagente. Nesse trabalho é utilizado um verificador de modelos para analisar algumas propriedades, dentre elas estão a verificação de ausência de deadlocks e/ou livelocks levando em consideração o tempo, pois o tempo é um fator essencial a ser considerado durante a verificação da simulação de futebol de robôs.  Por este motivo é utilizado o verificador de modelos chamado UPPAAL que oferece um conjunto de ferramentas para a verificação de sistemas em tempo real utilizando autômatos, além de utilizar uma linguagem chamada TCTL (Lógica de Árvore de Cálculo Temporizado) para formalizar as propriedades esperadas para o sistema. A diferença desse trabalho mostrado para esta proposta de pesquisa é que a verificação automática considera o tempo e sua abordagem foi desenvolvida especificamente para sistemas onde vários agentes atuam simultaneamente e por isso faz uso de uma outra ferramenta de verificação diferente da FDR.
	
Webster et al. (2014) propõe a verificação formal de robôs para assistência pessoal. Esses robôs estão presentes nas casas das pessoas e as ajudam em suas tarefas diárias, e como se trata de interação real com pessoas é necessária uma garantia de segurança, para que esses robôs não causem nenhum dano ou se coloquem em situações inesperadas. Por isso, é proposto um método de verificação formal. Essa proposta utiliza um verificador de modelos chamado SPIN que é bastante utilizado em sistemas de missões espaciais, telecomunicação e engenharias. A linguagem desse verificador é chamada PROMELA. Também foi utilizada uma linguagem chamada Brahms para modelar o robô. O processo de verificação ocorre quando o modelo escrito em Brahms é traduzido para linguagem PROMELA, onde o verificador de modelos pode verificar se as propriedades são satisfeitas ou não. Um exemplo de propriedade é saber se o robô irá se mover para cozinha quando o usuário envia o comando ao robô.  Essas propriedades são baseadas e pensadas na simulação e validação do usuário final, com o objetivo de aumentar a segurança prática e a confiabilidade dos assistentes robotizados, já que se trata de robôs físicos que estão trabalhando com pessoas reais. O processo de verificação é similar a esta proposta de pesquisa em alguns pontos, um deles é a tradução da linguagem modelada de um robô para uma especificação formal através de um compilador, contudo essa abordagem se trata de uma simulação voltada para o mundo real, enquanto a proposta deste trabalho é para ambientes educacionais de simulação de robôs virtuais, ainda mais que a DSL (Brahms) possui um propósito totalmente diferente da linguagem ROBO.

Outros trabalhos relacionados são: (KATS et al., 2012) onde é proposto Boogie, uma linguagem desenvolvida para gerar condições ou propriedades de verificação de outras linguagens de programação para que possam ser utilizadas por verificadores de programas da linguagem de origem; e (LEINO, 2008) no qual é discutido os desafios e oportunidades de pesquisa para a criação de ambientes de programação com a finalidade de disponibilizá-los na Web, através de um estudo de implementação de uma DSL e um ambiente de programação. Portanto, trabalhos relacionados existem, porém, não são voltados para a verificação automática de robôs educacionais de ambientes voltados para o aprendizado de programação.
Assim sendo, este trabalho de pesquisa é de grande contribuição para o contexto educacional. Uma vez que oferece para estudantes e professores, que utilizam ambientes de programação de robôs educacionais, um mecanismo de verificação automático efetivo para ajudá-los a obter feedback sobre a corretude dos seus programas escritos nesses ambientes. Além de contribuir também para área da Engenharia de Software através da verificação automática de sistemas educacionais utilizando uma abordagem de tradução automatizada. Portanto, esta proposta oferece grande relevância para a ciência.


\section{Trabalhos Futuros}

% ----------------------------------------------------------
% ELEMENTOS PÓS-TEXTUAIS
% ----------------------------------------------------------
\postextual
% ----------------------------------------------------------

% ----------------------------------------------------------
% Referências bibliográficas
% ----------------------------------------------------------
\bibliography{referencias.bib}

\appendix

\chapter{Módulos de ROBO - Gramática em SDF3}
\label{apendice1}
\lstinputlisting[caption=Módulo Common, label=cod:common]{codes/common.sdf3}
\fonte{O autor}

\lstinputlisting[caption=Módulo ExpressionsBoolean, label=cod:modbool]{codes/boolean.sdf3}
\fonte{O autor}

\lstinputlisting[caption=Módulo ExpressionsMath, label=cod:modmath]{codes/math.sdf3}
\fonte{O autor}

\chapter{Módulos de ROBO - Regras em Stratego}
\label{apendice2}
\chapter{Especificação CSP}
\label{apendice2}
%muito grande!!
\lstinputlisting[caption=Especificação CSP do problema Contando Caixas, label=cod:validation]{codes/program_validacao.csp}
\fonte{O autor}

%---------------------------------------------------------------------
% INDICE REMISSIVO
%---------------------------------------------------------------------
\phantompart
\printindex
%---------------------------------------------------------------------
\end{document}
